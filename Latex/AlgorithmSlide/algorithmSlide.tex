\documentclass[mathserif,notheorems]{beamer}

\usepackage[slantfont,boldfont]{xeCJK}
\usepackage{fontspec}
\setCJKmainfont{SimSun}
\setmainfont{SimSun}
\setsansfont{SimSun}
\setmonofont{Consolas}

\usepackage{amsfonts}
\usepackage{latexsym}
\usepackage{amsmath,amssymb,amsfonts}
\usepackage{color,xcolor}
\usepackage{graphicx}
\usepackage{algorithm}
\usepackage{subfigure}
\usepackage{amsthm}
\usepackage{listings}

% code setting
\newfontfamily\consolas{Consolas}
\lstset{ %
	%language=xml,
	basicstyle=\scriptsize\consolas,    % the size of the fonts that are used for the code
	% keywordstyle=\bfseries\color{black},
	% commentstyle=\itshape\color{black},
	keywordstyle=\bfseries\color{red!50!black},
	commentstyle=\itshape\color{green!50!black},
	stringstyle=\color{black},%{orange!80!black},
	numbers=left,                       % where to put the line-numbers
	numberstyle=\scriptsize,            % the size of the fonts that are used for the line-numbers
	%stepnumber=1,                      % the step between two line-numbers. If it is 1 each line will be numbered
	%numbersep=5pt,                     % how far the line-numbers are from the code
	backgroundcolor=\color{white},      % choose the background color. You must add \usepackage{color}
	xleftmargin=2em,
	xrightmargin=2em,
	showspaces=false,                   % show spaces adding particular underscores
	showstringspaces=false,             % underline spaces within strings
	showtabs=false,                     % show tabs within strings adding particular underscores
	frame=single,                       % adds a frame around the code
	tabsize=4,                          % sets default tabsize to 2 spaces
	captionpos=b,                       % sets the caption-position to bottom
	breaklines=true,                    % sets automatic line breaking
	breakatwhitespace=false,            % sets if automatic breaks should only happen at whitespace
	escapeinside={\%*}{*)},             % if you want to add a comment within your code
	inputpath={{snippet/}}
}

%\usetheme{Warsaw}
\usetheme{Madrid}
\usecolortheme{whale}

% ------------------------------------------------------------------

%\newtheorem{proposition}{Proposition}
\newtheorem{proposition}{命题}
\newtheorem{theorem}{定理}
\newtheorem{definition}{定义}
\newtheorem{example}{例}

\renewcommand\figurename{图}
\renewcommand\tablename{表}

\setbeamertemplate{theorems}[numbered]

% ------------------------------------------------------------------

\begin{document}

\title[算法设计]{算法设计实验课第一讲}
\author[Wsine]{韦政源}
\date[February, 2016]{2016~年~2~月~26~日}
% \institute{中山大学~移动信息工程学院}
% \titlegraphic{\includegraphics[height=0.15\textwidth]{SYSU_Logo.jpg}}
\frame{\titlepage}

\begin{frame}
\frametitle{算法设计实验课第一讲}
\linespread{1.2}   %%扩大行距 1.2 倍
\centerline{\bfseries\large\color{violet}题目}
\begin{enumerate}
\item 1020 Big Integer
\item 1020 Big Integer
\item 1020 Big Integer
\item 1020 Big Integer
\item 1020 Big Integer
\item 1020 Big Integer
\item 1020 Big Integer
\end{enumerate}
\end{frame}

\begin{frame} 
\linespread{1.2}
\frametitle{算法设计实验课第一讲}
\centerline{\bfseries\large\color{violet}分栏}
\begin{columns}
\column{.45\textwidth}
\begin{itemize}
\item DFS
	\begin{itemize}
	\item 1020 Big Integer
	\item 1020 Big Integer
	\item 1020 Big Integer
	\item 1020 Big Integer
	\item 1020 Big Integer
	\end{itemize}
\end{itemize}
\column{.45\textwidth}
\begin{itemize}
\item BFS
	\begin{itemize}
	\item 1020 Big Integer
	\item 1020 Big Integer
	\item 1020 Big Integer
	\item 1020 Big Integer
	\item 1020 Big Integer
	\end{itemize}
\end{itemize}
\end{columns}
\end{frame}

\defverbatim[colored]\lstI{
\begin{lstlisting}[language=C++]
int div(char x[], int b) {
	int a = 0; 
	// simulation computing
	for (int i = 0; x[i] != '\0'; i++) {
		a = (a * 10 + x[i] - '0') % b;
	}
	return a;
}
\end{lstlisting}
}
\begin{frame}
\frametitle{算法设计实验课第一讲}
\centerline{\bfseries\large\color{violet}1020 Big Integer}
\begin{description}
\linespread{2.5}   %%扩大行距 1.2 倍
\item[算法类别] 大数操作
\item[题目大意] 给出n个整数b$_1$,b$_2$,...,b$_n$,和一个大整数x,求x对每个数b$_i$取模的结果
\item[范围要求] n<=100, 1<b$_i$<=1000, x的长度不超过400
\item[解题思路] 对b$_i$逐个计算;高精度,模拟竖式计算
\end{description}
\lstI
\end{frame}

\begin{frame}
\frametitle{算法设计实验课第一讲}
\begin{exampleblock}{勾股定理}
$a^2 + b^2 = c^2$
\end{exampleblock}
\begin{block}{勾股定理}
$a^2 + b^2 = c^2$
\end{block}
\begin{alertblock}{勾股定理}
$a^2 + b^2 = c^2$
\end{alertblock}
\end{frame}

\begin{frame}
\frametitle{算法设计实验课第一讲}
\begin{table}[tb]
\centering
\caption{示例\label{tab:tablename}}
\begin{tabular}{l|cc} \hline
\textbf{column 1} & \textbf{column 2} & \textbf{column 3} \\ \hline
Hello & Beamer & NAN \\ \hline
$\alpha+\beta$ & $\gamma+\eta$ & 34\% \\ \hline
\end{tabular}
\end{table}
\end{frame}

\begin{frame}
\frametitle{算法设计实验课第一讲}
\begin{figure}[tb]
\centering
\includegraphics[width=0.9\textwidth]{sample.jpg}
\caption{示例\label{fig:figure1}}
\end{figure}
\end{frame}

\end{document}